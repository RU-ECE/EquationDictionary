\documentclass[12pt]{article}
\usepackage{amsmath}
\usepackage{authblk}

\title{Does the Inertia of a Body Depend Upon Its Energy Content?}
\author{Albert Einstein}
\date{Received 27 September 1905\\Published 21 November 1905, \emph{Annalen der Physik}, 18, 639–641}

\begin{document}
\maketitle

\section*{Introduction}
Building on earlier work using the Maxwell--Hertz equations (for empty space), Maxwell's expression for electromagnetic energy, and the principle of relativity, we deduce the following result:

A plane light‐wave system in frame $(x,y,z)$ has energy $\ell$, and propagates at angle~$\varphi$ to the $x$‐axis. In a frame $(\xi,\eta,\zeta)$ moving at velocity $v$ along $x$, this energy transforms to:
\[
\ell^* = \frac{\ell\,(1 - \tfrac{v}{c}\cos\varphi)}{\sqrt{1 - v^2/c^2}}.
\]

\section*{Energy Emission and Conservation}
Consider a stationary body in $(x,y,z)$ with energy $E_0$. In the moving frame $(\xi,\eta,\zeta)$ it has energy $H_0$. Suppose this body emits two equal plane light pulses of total energy $\ell$ in opposite directions making angles $\varphi$ and $\pi + \varphi$ relative to the $x$‐axis.

By conservation of energy in both frames:
\begin{align*}
E_0 &= E_1 + \tfrac12\ell + \tfrac12\ell = E_1 + \ell,\\
H_0 &= H_1 + \frac{\ell}{\sqrt{1 - v^2/c^2}}.
\end{align*}

Subtracting yields:
\[
(H_0 - E_0) - (H_1 - E_1) = \ell\left( \frac{1}{\sqrt{1 - v^2/c^2}} - 1 \right).
\]

\section*{Connection to Kinetic Energy}
Since $H$ and $E$ are energies of the same body in two inertial frames (one being at rest relative to the body), their difference can only differ by the kinetic energy $K$ in the moving frame up to a constant:
\[
H_0 - E_0 = K_0 + C,\qquad H_1 - E_1 = K_1 + C.
\]
Subtracting these eliminates $C$:
\[
K_0 - K_1 = \ell\left( \frac{1}{\sqrt{1 - v^2/c^2}} - 1 \right).
\]

\section*{Low-Velocity Approximation}
For small $v$, expand to second order:
\[
\frac{1}{\sqrt{1 - v^2/c^2}} - 1 \approx \frac12\frac{v^2}{c^2}.
\]
Hence,
\[
K_0 - K_1 \approx \frac12\frac{\ell}{c^2}\,v^2.
\]
But also,
\[
K_0 - K_1 = \tfrac12(m_0 - m_1)\,v^2.
\]
Equating gives the fundamental result:
\[
m_0 - m_1 = \frac{\ell}{c^2} \quad\Longrightarrow\quad \Delta m = \frac{\ell}{c^2}.
\]

\section*{Conclusion}
\begin{quote}
The mass of a body is a measure of its energy content; if the energy changes by~$\ell$, its mass changes in the same direction by $$\Delta m = \frac{\ell}{c^2}.$$
\end{quote}
In modern notation, replacing $\ell$ with $E$ and $m$ with $m$, we obtain the celebrated equation:
\[
E = m\,c^2.
\]

This remarkable equivalence implies that mass can be converted into energy and vice versa. Measuring the change in mass and multiplying by $c^2$ yields the energy emitted, and vice versa. Even small mass differences correspond to enormous energies, especially evident in radioactive processes such as with radium salts.

\end{document}
