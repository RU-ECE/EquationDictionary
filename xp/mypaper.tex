\usepackage[colorlinks=true,urlcolor=blue,hidelinks]{hyperref}
\documentclass[conference]{IEEEtran}
\IEEEoverridecommandlockouts
\usepackage{cite}
\usepackage{amsmath,amssymb,amsfonts}
\usepackage{algorithmic}
\usepackage{graphicx}
\usepackage{textcomp}
\usepackage{xcolor}
\usepackage{accessibility}
\def\BibTeX{{\rm B\kern-.05em{\sc i\kern-.025em b}\kern-.08em
    T\kern-.1667em\lower.7ex\hbox{E}\kern-.125emX}}


\begin{document}
\title{Information Theory and Channel Capacity: A Mathematical Analysis}

\author{\IEEEauthorblockN{Author Name}
\IEEEauthorblockA{\textit{Department of Electrical Engineering} \\
\textit{University Name}\\
City, Country \\
email@university.edu}
}

\maketitle

\begin{abstract}
This test paper presents equations, diagrams, and graphs.
The purpose is to define accessibility techniques. We are working on equations first.
Each equation will be linked to a web page where the symbols are defined,
\end{abstract}

\begin{IEEEkeywords}
information theory, channel capacity, discrete channels, Shannon theory, mathematical communication
\end{IEEEkeywords}

\section{Introduction}
\begin{equation}
\rho \left(\frac{\partial \mathbf{v}}{\partial t} + \mathbf{v} \cdot \nabla \mathbf{v}\right) = -\nabla p + \mu \nabla^2 \mathbf{v} + \mathbf{f}\hspace{1em}\href{http://localhost:8000/eqn/fd.Navier-Stokes}{\underline{lookup}}
\label{eq:fd.Navier-Stokes}
\end{equation}
 % insert the equation, link to the web page with the dictionary entry

\begin{equation}
\sin^2 x + \cos^2 x = 1\hspace{1em}\href{http://localhost:8000/eqn/ident.Pythagorean-Identity}{\underline{lookup}}
\label{eq:ident.Pythagorean-Identity}
\end{equation}


\begin{thebibliography}{1}
\bibitem{shannon1948}
C. E. Shannon, ``A mathematical theory of communication,'' \textit{The Bell System Technical Journal}, vol. 27, no. 3, pp. 379--423, 1948.
\end{thebibliography}

\end{document}
